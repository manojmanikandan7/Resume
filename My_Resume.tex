% Document class and font size
\documentclass[a4paper,1.5pt]{extarticle}

% Packages
\usepackage[utf8]{inputenc} % For input encoding
\usepackage{geometry} % For page margins
\geometry{a4paper, margin=0.4in} % Set paper size and margins
\usepackage{titlesec} % For section title formatting
\usepackage{enumitem} % For itemized list formatting
\usepackage{hyperref} % For hyperlinks
\hypersetup{
    pdftitle={Manoj Manikandan Resume},
    colorlinks=true,
    linkcolor=blue,
    filecolor=magenta,
    urlcolor=blue,
    pdftitle={Resume - Manoj Manikandan},
    }


% Formatting
\setlist{noitemsep, leftmargin=12pt, rightmargin=15pt, topsep=1.25pt} % Removes item separation
\titleformat{\section}{\large\bfseries}{\thesection}{1pt}{}[\titlerule] % Section title format
\titlespacing*{\section}{0.5pt}{*0.75}{*0.75} % Section title spacing

% Begin document
\begin{document}

% Disable page numbers
\pagestyle{empty}

% Header
\begin{center}
\textbf{\Large MANOJ MANIKANDAN}\\[2pt] % Name
\href{mailto:manojmani1408@gmail.com}{manojmani1408@gmail.com} | \href{tel:447543823791}{+447543823791}  | \href{https://www.linkedin.com/in/manojmanikandan}{linkedin.com/in/manojmanikandan}  |  \href{https://github.com/manojmanikandan7}{github.com/manojmanikandan7}
\end{center}
% Education Section
\section*{EDUCATION}
\textbf{The University of Manchester} | Manchester, United Kingdom \hfill Sep 2023 | Jun 2026\\ % University name and location 
\textit{Bachelor of Science (Honours) Computer Science} \hfill \textbf{Expected: First-Class Honours} % Degree and GPA 
\begin{itemize}
    \item \textbf{Year 2: } \textbf{81\%}, \textbf{Top 15} in a class of \textbf{300}.
    \item \textbf{Courses:} Software Engineering, Data Structures and Algorithms, Artificial Intelligence, Machine Learning, \\
    Distributed Systems, Database Systems, Processor Micro-architecture, Operating Systems. %Computer Architecture, Computer Engineering, 
    % \item Student Representative and Ambassador for the Department of Computer Science.
\end{itemize}

 %\hfill Credit standing: Honors % Additional info
% \noindent
% \textbf{JSS INTERNATIONAL SCHOOL} | Dubai, United Arab Emirates \hfill Jun 2015 | Mar 2023\\ % University name and location
% \textit{High School Diploma - affiliated with CISCE} \hfill \textbf{Grade: 97\%} % Degree and GPA


% Experience Section
\section*{EXPERIENCE}
% \textbf{University of Manchester} \\ %\hfill City, State\\ % Company name and location
% \textit{Technology Intern} \hfill Apr 2023 | Aug 2023 % Position and duration
%\begin{itemize}
    %\item Attained insight on Generative AI, utilising LLMs provided by OpenAI and Meta.  % Job responsibilities and achievements
    %\item Designed a working chatbot to generate around 18 types of HR documents, planned to be incorporated in the system.
    %\item Increased the efficiency of running basic operations around the workspace by 50\%.
    %\item Acquired skills in code documentation, presentation, testing and technical writing.
    %\item Mentored under a Machine Learning expert to complete my project.

%\end{itemize}
\textbf{N8 Computationally Intensive Reasearch} \hfill Jul 2025 | Aug 2025 \\ 
\textit{Research Software Engineer Intern} \hfill Manchester, United Kingdom \\ % Company name and location\\ % Position and duration 
\href{https://n8cir.org.uk/themes/internships/internships-2025/manchester/}{Profile Link}
\begin{itemize}
    \item Collaborated with \textbf{7 academics} to drive digitization initiatives in humanities fields, spearheading \textbf{6 projects}.
    \item Remodeled the codebase of the \href{www.spokencorpus.org}{American Spoken Corpus} site, reducing complexity by \textbf{50\%}. Implemented a \href{https://spokencorpus.org/map_search.php}{Dialect Map}, for visualisation of over \textbf{150 transcripts}.
    \item Extracted and visualized migration patterns from over \href{https://conflictmemorymigration.manchester.ac.uk/interviews/}{\textbf{70 interview transcripts}} of the Troubles' victims. 
    \item Devised a \href{https://github.com/manojmanikandan7/WikipediaXmlParser}{processing tool} to pre-process, extract and collate a corpus of around \textbf{600 raw text documents} from Wikipedia XML exports. 
\end{itemize}
\textbf{HRBluSky, Pruvity Group} \hfill Apr 2023 | Aug 2023 \\ %\hfill City, State\\ % Company name and location
\textit{Technology Intern}  % Position and duration
\begin{itemize}
    \item Mentored under an ML expert to complete the project, gaining experience in code structuring, documentation, testing and technical writing.
    \item Evaluated performance of fine-tuned Language Models on templated text generation, streamlining the process of artificial document generation for HR by \textbf{65\%}.
    \item Assembled an AI-based assistant to generate around \textbf{18 types} of HR documents, resulting in improvement of running basic operations around the workspace by \textbf{30\%}.
\end{itemize}


% Additional Experience or Volunteer Work
%\noindent
%\textbf{Project or Volunteer Work Name} %\hfill City, State\\ % Project or organization name and location
%\textit{Position Title, Volunteer} \hfill Month Year – Month Year % Position and duration
%\begin{itemize}
    %\item Description of responsibilities and achievements at this position. % Responsibilities and achievements
%\end{itemize}

% Club or Organization Experience
%\noindent
%\textbf{University of Manchester} \hfill Manchester, United Kingdom\\ % Club or organization name and location
%\textit{Student Representative} \hfill Oct 2023 – Ongoing % Position and duration
%\begin{itemize}
    %\item Description of responsibilities and achievements at this position. % Responsibilities and achievements
%\end{itemize}

% Skills Section
\section*{SKILLS}
\begin{itemize}
    \item \textbf{Programming and tools:} Java, Spring Boot, Python, Rust, C, C++, PHP, ReactJS, JavaScript, MySQL, TypeScript, HTML, CSS, ARMv7 assembly, Verilog, SystemVerilog.
    \item \textbf{Technical:} Deep Learning, Convolutional Neural Networks, Recurrent Neural Networks, LSTMs, Language Models, Transformers, Transfer Learning, Prompt Engineering, Encryption Algorithms. 
    % Software skills % Communication skills
\end{itemize}

% Projects Section
\section*{PROJECTS}
\noindent
\textbf{Interactive Polish Magazine Transcriber} \hfill Aug 2025 \\ % \hfill City, State\\ % Project name and location
\href{https://github.com/manojmanikandan7/PolishMagTextRecog}{Link}  % Project link and duration
\begin{itemize}
    \item Engineered an interactive transcriber for Polish magazine scans, based on Google's \texttt{tesseract} transcription engine.
    \item Processed over \textbf{215 pages} of Polish Queer magazines and compiled their transcripts.
    \item Retrieved around \textbf{10,000 blocks} of texts, utilised in further analysis and study.
 % Project description and contributions
\end{itemize}

% Team Project
\noindent
\textbf{PennyWise - Student-friendly Budget Tracker} \hfill Sep 2023 | May 2024 \\ % Project name and location
\href{https://github.com/manojmanikandan7/PennyWise}{Link}
\begin{itemize}
    \item Scored \textbf{92\%} for the overall project grade.
    \item Collaborated with a team of \textbf{6 students} to develop a budget tracker for students, as part of the First-Year Team Project at University of Manchester.
    \item Presented the project to around \textbf{100 students}.
    \item Improved awareness of budgeting for around \textbf{50 university students}.
 % Project description and contributions
\end{itemize}

% Internship Project
% \noindent
% \textbf{LetterBot - AI based Chat-bot to produce HR Documents} \hfill Apr 2023 | Aug 2023 \\ % Project name and location
% \textit{Technology Intern, HRBluSky}  % Project link and duration
% \begin{itemize}
%     \item An AI-based chat-bot to create HR documents curated to inputs by the user.
%     \item Streamlined the process of artificial document generation by 65\%.
%     \item Incorporated the OpenAI API in Python to develop the chat-bot.
%     \item Implemented preprocessing of the commands and formatting the generated documents.
%     \item Produced a custom fine-tuned model based on the GPT-3 from OpenAI.
%  % Project description and contributions
% \end{itemize}

% Personal Project
% \noindent
% \textbf{Encryption of data using the AES encryption algorithm in Java} \hfill Jan 2022 | Mar 2022 \\ % \hfill City, State\\ % Project name and location
% \href{https://github.com/manojmanikandan7/Cryptography}{Link}  % Project link and duration
% \begin{itemize}
%     \item Focused on encrypting data using the Advanced Encryption Standard (AES), a symmetric block cipher. 
%     \item Integrated a complementary decryption program which takes in the hashed data and private key to decrypt the data.
%  % Project description and contributions
% \end{itemize}




\section*{LEADERSHIP AND AWARDS}
\noindent
\textbf{University of Manchester} | Manchester, United Kingdom \hfill Sep 2023 | Jun 2026 % Club or organization name and location
\begin{itemize}
    \item Attained the prestigious Manchester International Excellence Scholarship 2023, presented to only \textbf{50 students} out of around \textbf{3000 candidates}.
    \item Contracted and supervised \href{https://www.greatunihack.com}{GreatUniHack} 2024, Manchester, with around 200 participants.
    \item Nominated as a Student Representative for a class of around \textbf{300} Computer Science students.
    \item Peer Mentor (PASS) for \textbf{20 underclassmen} to provide academic and well-being support.
 % Responsibilities and achievements
\end{itemize}

\noindent
\textbf{JSS INTERNATIONAL SCHOOL} | Dubai, United Arab Emirates \hfill Jun 2015 | Mar 2023
\begin{itemize}
    \item Received Proficiency awards for scoring above \textbf{90\%} overall from Years 9 to 12.
    \item Co-founder and Tech support at CollegePrep101. Raised college preparedness among students by about \textbf{40\%}.
\end{itemize}


% Certifications
\section*{CERTIFICATIONS}
\noindent
\textbf{Deep Learning Specialization} \hfill Aug 2024 \\
\href{https://www.coursera.org/account/accomplishments/specialization/LR4ZVQP24B9B}{Credential Link}
\begin{itemize}
    \item \textbf{Courses Completed:} Neural Networks and Deep Learning, Improving Deep Neural Networks: Hyperparameter Tuning, Regularization and Optimization, Structuring Machine Learning Projects, Convolutional Neural Networks, Sequence Models.
    
\end{itemize}


% End document
\end{document}
